\documentclass{sistedes}

\begin{document}
\def\uschema{\leavevmode{\hbox{U-Schema}}}

\title{A Generic Schema Evolution Approach for NoSQL and Relational Databases\thanks{This work has been funded by the Spanish Ministry of
  Science and Innovation (project grant PID2020-117391GB-I00).}}

\author{
Alberto~Hernández~Chillón\inst{1}\orcidID{0000-0002-1154-9192}
\and
Meike~Klettke\inst{2}\orcidID{0000-0003-0551-8389}
\and
Diego Sevilla Ruiz\inst{1}\orcidID{0000-0001-9313-008X}
\and
Jesús~García~Molina\inst{1}\orcidID{0000-0003-4685-6659}
}

\institute{
Faculty of Computer Science,
    University of Murcia, Murcia, Spain\\
\email{\{alberto.hernandez1,dsevilla,jmolina\}@um.es}\\
\and
Faculty of Computer
    Science and Data Science, University of Regensburg, Regensburg,
    Germany\\
\email{meike.klettke@ur.de}
}

\maketitle

\small

\keywords{NoSQL databases, Schema evolution, Evolution management, Taxonomy of
  changes, Schema change operations}


%% Specify where this paper was published

% For journal papers, uncomment the command below indicating the journal name, volume, issue, pages and year
\publishedin{IEEE Transactions on Knowledge and Data Engineering,
  Vol.~36, Issue~7, pp.~2774--2789, July~2024}

% For conference papers, uncomment the command below indicating the
% acronym of the conference, year, proceedings name, and pages
% \publishedin{Acronym YEAR - Proceedings of full conference name, pp.
% XXX--YYY, YEAR.}

%% Specify impact information of your publication. Below, you can find
%% some templates that may be used to justify the impact of your
%% publication If none of the templates below fits your needs, you may
%% want to include any other indicator that allows verifying the
%% quality of your work according to the CoARA principles
%% (https://coara.eu/coalition/guiding-principles/)

% For journal papers, uncomment the command below indicating the JCR
% IF, quartile, position, and area
\impact{JCR Q1 - Area: ENGINEERING, ELECTRICAL \& ELECTRONIC - (IF: 8.9)}

% For conferences, uncomment the command below indicating the GGS
% class\impact{GII-GRIN-SCIE Class X (Rating)}

% If your conference does not require to provide impact information,
% you can just remove the the \impact{...} commands above


\DOI{https://doi.org/10.1109/TKDE.2024.3362273}


\begin{abstract}
  In the same way as with relational systems, schema evolution is a
  crucial aspect of NoSQL systems. But providing approaches and tools
  to support NoSQL schema evolution is more challenging than for
  relational databases. Not only are most NoSQL systems schemaless,
  but different data models exist without a standard specification for
  them. Moreover, recent proposals fail to address some key aspects
  related to the kinds of relationships between entities, the
  definition of relationship types, and the support of structural
  variation.

  In this article, we present a generic schema evolution approach able
  to support the most popular NoSQL data models (columnar, document,
  key-value, and graph) and the relational model. The proposal is
  based on the Orion language that implements a schema change
  operation taxonomy defined for the \uschema{} unified data model
  that integrates NoSQL and relational abstractions. The consistency
  of the taxonomy operations is formally evaluated with Alloy, and the
  Orion semantics is expressed by translating operations into native
  code to update data and schema. Several database systems are
  supported, and the engine built for each of them has been validated
  by testing each individual SCO and refactoring study cases. A study
  of relative execution time of operations is also shown.
\end{abstract}

\end{document}
